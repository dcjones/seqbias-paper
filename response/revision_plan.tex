
\documentclass{article}
\usepackage{dcj}
\usepackage{ulem}

\begin{document}

\dcjtitle{Seqbias Paper}{Plan for Revision}{Daniel Jones}

\begin{enumerate}

\item Demonstrated the results are not particularity sensitive to the
complexity penalty, drawing a plot similar to the one is Supplementary Section
3, but varying the penalty rather than the number or reads.

\item Point out that the fact that the method is effective over diverse sets of
data argues that the results are not sensitive to any of the parameters used.

\item Note that the Hansen's brute-force method was trained using many more than
600,000 reads on some of the data sets.

\item (Maybe) Draw a curve showing how the performance of Hansen's brute-force
method changes as a function of the number of reads fed into it.

\item Make very clear that in there is no bias present in the data, a sparse or
empty model will be trained.

\item Include a figure from the genome browser showing an example uniformity of
coverage being increased over a gene or exon.

\item Use the MAQC set to demonstrate the following:
\begin{enumerate}
\item Increased agreement between bias corrected sequencing data and qPCR.
\item Increased agreement between replicates.
\end{enumerate}

\item \sout{Examine the bias in new methods without a PCR-amplification step.}

\item Typos mentioned by revier 1.


\end{enumerate}


\end{document}

